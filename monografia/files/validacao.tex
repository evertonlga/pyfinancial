\chapter{Verificação e Validação}
Uma das etapas mais importantes dos processos de desenvolvimento de software é aquela referente as atividades de Verificação e Validação. A maioria dos processos modernos, inclusive XP1, indica que as atividades de teste devem ser executadas paralelamente a codificação do sistema, ou seja, existindo código já deve existir testes para o mesmo. Tal atitude garantirá que os testes do sistema possam evoluir conjuntamente com o crescimento do volume de código.

Pelo contexto do nosso projeto estar baseado principalmente na utilização de fórmulas matemáticas e financeiras, procuramos fundamentar bem nossos resultados através de uma bateria, grande em volume e em qualidade, de testes (testes de unidade e de aceitação). Para gerenciamento e utilização sistemática dessa bateria de testes fizemos uso de bibliotecas para execução automática dos mesmos. As bibliotecas utilizadas foram: PyUnit para testes de unidade e PyEasyAccept para testes de aceitação.

Outra preocupação constante durante a construção dos testes de nossa aplicação foi com o nível de tolerância que os resultados obtidos poderiam divergir dos resultados apresentados pela HP-12C. Este foi um dos requisitos não-funcionais planejados e, para cumprí-lo, fez-se uso em todos os testes de unidade de uma função especial para averiguar que os resultados dos testes estivessem dentro da tolerância pré-estabelecida. Quanto aos testes de aceitação verificou-se se os resultados que estavam sendo retornados estavam dentro da faixa desejada.

Para validação da aplicação, durante as reuniões quinzenais com o cliente do projeto e com o professor Adail Marcos, que leciona a disciplina de Matemática Financeira no curso de Administração da UFCG e que nos orientou em relação as fórmulas a serem empregadas, pudemos validar se questões como interface e novas funcionalidades estavam de acordo com as expectativas dos possíveis clientes do nosso produto final.

Toda esta preocupação com testes, além de objetivar minimizar a quantidade de falhas do produto, tiveram por propósto atestar a qualidade do nosso software, bem como permitir que a evolução do mesmo possa ser realizada sem maiores problemas.

