\chapter{Novas Funcionalidades}

Um das características de um software bem projetado é permitir facilmente a inclusão de novas funcionalidades. 
Aqui será apresentado os passos necessários para a inclusão de novas funcionalidades.

Antes de apresentar o passo-a-passo, deve-se levantar as seguintes considerações:

\begin{enumerate}
 \item Em relação a interface gráfica, o módulo principal que organiza os dados de interface gráfica da calculadora, 
incluindo disposição de botões, é o CalcGui. Além desse, existem módulos que lidam com a tabela de amortização 
(calcTabelDialog), com a tabela de recall (recallViewDialog), com a tabela de store (storeViewDialog) e a 
representação genérica de um botão (calcButton).

\item Existe um componente que representa um botão (calcButton), podendo esses ter adicionados até 3 textos de 
funções: uma na parte superior, um na parte inferior e um na parte central.

\item O controller (calcController) captura todos os eventos de teclas pressionadas, considerando o texto da parte central do botão 
(que representa a função principal de cada tecla). Em seguida chama uma ação correspondente no model (calcModel)

\item O calcModel é a facade do model do sistema e guarda o estado da calculadora (opções de arredondamento, 
se as teclas especiais f ou g estão pressionadas, etc) e conhece a pilha (calcStack), os demais registradores (calcRegs) 
e as funções da biblioteca financeira (pyFinancialLibrary).

\item As várias funções presentes em calcModel representam chamadas as teclas de uma dada função principal 
(a funcão cujo texto fica no meio do botão). Dado que o model conhece o estado da calculadora, em cada uma dessas 
chamadas verifica-se que função da biblioteca usar verificando o estado de F e G.

\item Na ligação do modelo, da interfaçe gráfica e do controlador foi utilizado o modelo MVC \cite {mvc}. 
Os módulos da interface gráfica funcionam como observers. O model tem vários listeners para eventos como: 
modificação do valor da tela, modificação da pilha, etc. Ao final de cada função do model, faz-se as ações necessárias 
(desativar F ou G pressionados; avisar listeners sobre alterações na pilha, registradores financeiros, etc, 
dependendo do que a função faça; alertar a interface sobre modificação na tela, no estado dos botões ou sobre 
a necessidade de mostrar a tabela de amortização, de recall ou de store.

\item Os dados fornecidos de volta aos listeners estão sempre no formato de Decimal, que é o tipo trabalhado pela 
calculadora de modo a alcançar alta precisão. Cada módulo da GUI formata esses dados como queira, usando os métodos 
de formatacão do módulo calcFormatUtil.

\end{enumerate}

Agora, como exemplo, irá se utilizar a implementação de uma nova forma de divisão em que ao invés de se dividir o 
registrador Y pelo X, irá se dividir X por Y. Através desse exemplo, pretende-se abordar todo o passo-a-passo para se 
implementar uma nova funcionalidade.

Primeiramente é preciso escolher um botão do teclado da GUI que esteja livre para receber a nova função. 
No método \begin{verbatim} __createKeyboardLayout \end{verbatim}
do módulo calcGui é feita a adição do botões na tela. Escolha uma posição e adicione o 
botão através de \begin{verbatim}self.__addGridButton(grid, "INV /", 1, 5)\end{verbatim}

Agora, criar a ação relativa ao botão no método \textbf{keyboardButtonClicked(self, text)}
do controller: 

\lstinputlisting[language=Python, label=else, caption={Teste no controller}]{else.py}
% \begin{verbatim}
% 	elif text == "INV /":
%  		self.__model.invDiv()
% \end{verbatim}

No model, criar o método chamado no controller. Nele, implementar a nova funcionalidade baseado nos valores de 
F e G, nos valores dos registradores e no estado da calculadora. Com o resultado da divisão em mãos, setar algum dos 
registradores conforme necessário. Desativar as teclas F e G, e conforme o que for modificado (tela, pilha, registradores 
financeiros, etc) avisar os devidos listeners sobre as alterações. Um ponto importante é também alterar o modo de 
entrada de dados da calculadora de acordo com a funcionalidade desejada. Existem basicamente três modos: 
Mode.SaveMode (que devolve o valor 0.0 a tela), Mode.OperationMode (devolve zero a tela, mas antes dá um enter na pilha) 
e Mode.EntryMode (que indica que números estão sendo entrados na tela).

\lstinputlisting[language=Python, label=funcao, caption={Definição de Função no Model}]{funcao.py}


% \begin{verbatim}
% 
%     def invDiv(self):
%         if self.__isGPushed or self.__isFPushed:
%             self.__fireException("Function not yet implemented.")
%             self.__deactiveFandG()
%         else:
%             # "inv /"
%             try:
%                 result = self.__calcStack.getXReg() / self.__calcStack.getYReg()
%                 self.__calcStack.rollCounterClockWise()
%                 self.__calcStack[0] = result
%                 self.__calcStack[3] = self.__calcStack[2]
%         
%                 self.mode = Mode.OperationMode
%                 self.base = None
%                 self.dotActived = False
%         
%                 self.__fireStackRegisters(self.getAllStackRegisters())
%                 self.__fireScreenChangedEvent()
%                  
%             except Exception, e:
%                 self.__fireException(e.message)
% \end{verbatim}
