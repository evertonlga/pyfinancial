\chapter{Introdução}

% * Descrição do problema (Business case: Por quê? Para quem? Problemas que originaram a necessidade pela nova aplicação): http://www.valoreducacao.com.br/home/a-empresa/38-gerais/48-planejamento-financeiro-pessoal, 
% http://www.matematica.ucb.br/sites/000/68/00000039.pdf, Livro matematica financeira (ato uso das calculadoras hj em dia)

% * Visão da solução (visão caixa-preta da aplicação e seus principais objetivos).  

% * Contexto do Projeto. Pessoas e Processos (Perfil dos Usuários/Skateholders e Processos que vão empregar o sistema), Ambiente de Execução (Características Fundamentais e Restrições), Aplicações Correlatas (Sistemas Legados, Outras Aplicações Semelhantes, Sistemas e Processos em Vigor)

Muito tem se discutido hoje em dia a respeito da necessidade de um melhor gerenciamento de finanças pessoais por parte dos membros da sociedade. Cada vez mais as pessoas percebem como é importante administrar bem seus recursos financeiros de modo a possibilitar o alcance de várias metas como, por exemplo, aquisição de imóveis, veículos, móveis de melhor qualidade ou até mesmo a aposentadoria. Outro ponto bastante importante é que munido dos conhecimentos básicos, a pessoa torna-se um consumidor muito mais responsável e apto a lidar com o mercado, bem como a realizar investimentos. Porém, em especial na sociedade brasileira, fala-se muito a respeito de possibilidades de investimento sem que antes a população receba uma educação financeira adequada \cite{valoreducacao}. 

O tópico de matemática financeira é comumente lecionado nas escolas e análises a res\-peito de como esses cursos estão sendo conduzidos podem ser encontradas em \cite{educacaoMedio}. Porém, ao analisarmos o comportamento das pessoas no dia a dia do comércio são poucos os que têm uma noção clara dos princípios financeiros e que consequentemente os usam rotineiramente. Logo, o uso de calculadoras financeiras facilitam o aprendizado e o exercício dos conceitos e pode auxiliar nos cálculos do cotidiano. Além desse ponto, os especialistas confirmam que a matemática financeira não é mais praticada atualmente sem o uso de tais dispositivos \cite{matFinanceira}, em especial devido à complexidade de certos cálculos que tomariam muito tempo para serem desenvolvidos manualmente.

Nesse contexto, a HP-12C \cite{hp12c} se coloca como uma calculadora líder de mercado há vários anos, inclusive com vários modelos que foram desenvolvidos ao longo desse tempo, e muitos são os cursos e livros que ensinam Matemática Financeira com o apoio da mesma. É fato que para profissionais da área como Administradores, Economistas, dentre outros, a mesma se apresenta como uma boa calculadora e bastante confiável. Entretanto, ao ana\-lisar usuários que usam alguns dos conceitos financeiros mas não são especialistas é comum observar-se uma certa rejeição a este tipo de calculadora, devido à complexidade para realizar certas análises como, por exemplo, a de um plano de amortização. Um produto que seja mais prático facilitaria o aprendizado e uso desses conceitos.

Observando um pouco a história recente dos dispositivos móveis percebe-se o grande crescimento do uso dos mesmos no cotidiano das pessoas \cite{celular}. Conjuntamente com esse aumento de uso/venda, surgiu também um forte mercado para o desenvolvimento de aplicativos para dispositivos móveis e hoje são inúmeros os softwares que podem ser adquiridos para as mais variadas plataformas, tanto através de compra nos sites oficiais de fornecedores, de desenvolvedores autônomos ou ainda através da pirataria. Esses aplicativos tipicamente perpassam o mundo da multimídia, como jogos, players, etc., ao da organização de tarefas, com calendários e agendas, dentre outros.

Entretanto, não é comum vermos aplicativos relacionados ao mundo financeiro. Recentemente um grande avanço nessa área foi o lançamento de um emulador da HP-12C para o Iphone da Apple \cite{iphone}. Logo, observa-se aqui uma boa oportunidade de alavancar um pouco mais esse nicho. Para tal, firmou-se como objetivo do projeto o desenvolvimento de dois artefatos: i) uma biblioteca de funções, escritas em Python, que implementem as principais funções financeiras existentes na HP-12C. Com tal resultado obtém-se um auxílio para o desenvolvimento de novos aplicativos na área financeira, dado que, de acordo com as pesquisas realizadas, percebemos a inexistência de pacotes de software que forneçam certas operações básicas que comumente estão presentes nas várias aplicações financeiras; e ii) uma calculadora financeira para os dispositivos da Nokia, mais especificadamente para o N800, a qual foi designada de \textit{PyFinancial}, que terá como objetivo implementar de forma mais amigável, intuitiva e completa os requisitos essenciais para cálculos financeiros. 

Do ponto de vista funcional, a calculadora \textit{PyFinancial} é um software com propósitos financeiros que irá oferecer ao usuário o cálculo das várias funções financeiras básicas existentes na HP-12C, como análises através de juros simples e compostos, bem como cálculo de planos de amortizações.  A solução proposta visa oferecer um nível de confiança similar ao da HP-12C, com resultados bem próximos aos que são encontrados na mesma, e, tirando proveito das facilidades de interação que o dispositivo nos provê, oferecer uma maior praticidade no cálculo das várias funções financeiras ali presentes. A interface \textit{touch-screen} que o dispositivo oferece é uma dessas facilidades que torna o uso da calculadora mais prático.


\section{Contexto do Projeto}

\subsection{Pessoas e Processos}
% * Pessoas e Processos (Perfil dos Usuários/Skateholders e Processos que vão empregar o sistema)

Conforme relatado acima, o perfil principal de usuários a serem alcançados com a aplicação são aqueles que não possuem conhecimentos aprofundados em matemática financeira e que consequentemente necessitam de uma interface mais amigável para a realização dos cálculos.

Entretanto, não pode-se descartar, devido à facilidade de interação buscada, aqueles usuários que desejam aprender a respeito de matemática financeira. Esses podem encontrar na calculadora uma boa oportunidade para de fato realizar seu aprendizado, e esse grupo de usuários vem tornando-se crescente, conforme já relatado anteriormente. Além destes, pode-se conseguir atrair também aquelas pessoas que fazem uso desses cálculos cotidianamente, mas que acham complicado a execução de certas operações nas calculadoras existentes no mercado. Para tanto, buscou-se opiniões com profissionais e alunos da área financeira de modo a aglutinar idéias de bons formatos de interação entre o usuário e a aplicação.


\subsection{Ambiente de Execução}

% * Ambiente de Execução (Características Fundamentais e Restrições),

Para a implantação da aplicação faz-se necessário que o usuário possua:

\begin{itemize}
 \item Dispositivo Internet Tablet N800 da Nokia \cite{n800}. Atualizações do código para o dispositivo N810 estão no planejamento de evolução da calculadora.
 \item Sistema operacional Maemo Diablo (4.1.x) \cite{diablo}.
 \item Ambiente gráfico QT4 \cite{qt4} e PYQT4 \cite{pyqt4}
 \item Python 2.5 \cite{python}.
\end{itemize}

É importante destacar que não foram realizados testes com nenhuma versão posterior as que foram acima relatadas. O motivo principal para a não realização desses testes é a instabilidade de algumas dessas versões.


\subsection{Aplicações Correlatas}

% * Sistemas Legados, Outras Aplicações Semelhantes, Sistemas e Processos em Vigor

Nesta seção será relatado um pouco a respeito de algumas aplicações similares a calculadora \textit{PyFinancial}:

\begin{itemize}

 \item \textit{Mobile Financial Calculator V1.0} \cite{mobcalc}: Essa calculadora se assemelha a aplicação proposta na facilidade de uso, utilizando-se de vários menus, e oferece um conjunto variado de funções. Porém, destina-se a dispositivos um pouco mais antigos, por exemplo dispositivos Série 60 \cite{s60}, o que não lhe permite um alto grau de interação, como o proposto pela \textit{PyFinancial} através da interface \textit{touch-screen} do N800.

 \item \textit{HpCalc-Iphone} \cite{hpiphone}: É um emulador da calculadora HP-12C disponibilizado para o dispositivo Iphone da Apple. Embora forneça um alto grau de interação através da interface \textit{touch-screen}, a mesma, por ser um emulador, é uma cópia fiel da HP-12C, o que implica na permanência das dificuldades de uso para usuários menos habituados com os conceitos financeiros.

 \item \textit{Web HP-12C emulator} \cite{epxcalc}: É um emulador da HP-12C oferecido na Internet que, de maneira similar ao emulador anterior, possui as mesmas dificuldades de uso já relatadas, mas possui um problema mais grave que diz respeito à acurácia de seus resultados. Comparou-se resultados oferecidos pela mesma com resultados da HP-12C original e foram percebidas distorções consideráveis em certo conjunto de cálculos, por exemplo, nos cálculos do pagamento e número de períodos.

 \item \textit{Finance Calculator Version 4.2} \cite{arachnoid}: É uma calculadora financeira oferecida na Internet desenvolvida em JavaScript e que possui um alto grau de facilidade em seu uso, bem como uma alta acurácia nos resultados apresentados em comparação com a HP-12C original, inclusive com documentação das fórmulas usadas. A grande fraqueza da mesma é o fato de disponibilizar apenas os cálculos dos valores financeiros básicos como valor presente, valor futuro, pagamento, número de períodos e taxa de juros.

\end{itemize}

