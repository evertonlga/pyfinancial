\chapter{Conclusões e Trabalhos Futuros}

Um dos méritos desse projeto foi trabalhar com conhecimento multidisciplinar, envolvendo 
conhecimentos adquiridos em disciplinas de computação, como Engenharia de Software, Métodos de
Software Numérico e Sistemas de Informação I e II, assim como em disciplinas extras, como
Princípios de Administração Financeira. Dessa forma, os integrantes da equipe tiveram a
oportunidade de exercitar os conhecimentos adquiridos em sala de aula.

A opção por desenvolver uma aplicação para dispositivos móveis também contribuiu para o amadurecimento
da equipe no tocante a perceber as particularidades e as dificuladades de se trabalhar no desenvolvimento desse tipo
de sistema. Além disso, essa experiência possibilitou o contato com diversas novas tecnologias,
aumentando o conhecimento da equipe e inserindo diferenciais nos currículos profissionais.

Uma das dificuldades enfrentadas pelo grupo foi a falta de apoio na área financeira. O cliente
cumpriu muito bem o seu papel, porém este tinha maior embasamento na área de dispositivos móveis.
Houveram diversas tentativas, por parte do grupo e do cliente, em obter um apoio na área financeira,
porém com pouco sucesso.

Como trabalhos futuros, planejamos a adição de novas fórmulas financeiras, bem como o refatoramento
das fórmulas existentes para introduzir métodos melhores para calcular os resultados. Planeja-se
também trabalhar em melhorias na interface com o usuário, visto que isso não foi possível no
decorrer das disciplinas pois a biblioteca gráfica utilizada ainda estava em processo de migração
para a plataforma Nokia Maemo.

Enfim, toda a experiência vivida em Projeto I e II possibilitou o crescimento dos alunos como
desenvolvedores, analistas e gerentes de projeto, aguçando as habilidades de resolução
de problemas e permitindo o contato com novos conceitos e paradigmas.
