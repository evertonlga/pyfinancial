\chapter{Glossário} \label{glossario}

\textbf{A}

\begin{itemize}
\item Amortização:
    A palavra amortização pode ser entendida como a diminuição da dívida, de uma única vez ou aos poucos, em mais vezes. 
\end{itemize}

\textbf{C}
\begin{itemize}
\item Capitalização: 

    Periodicidade do vencimento do juro ou número de vezes em que o juro é processado (calculado) num ano: anual, semestral, trimestral, mensal, etc. 

\item Capitalização Linear:
    A capitalização linear – ou método de formação dos juros simples – considera a incidência periódica de uma dada taxa de juros sobre uma base de cálculo fixa. 

\item Capitalização exponencial:
    A capitalização exponencial ou – método de formação dos juros compostos – considera a incidência periódica de uma dada taxa de juros sobre uma base de cálculo variável. 

\item Capital inicial:
    Capital existente na data-pólo inicial. 

\item Capital final:
    Capital existente na data-pólo final. 

\end{itemize}

\textbf{D}
\begin{itemize}
 \item Data-pólo inicial:
    Data de início do investimento. 

\item Data-pólo final:
    Data de resgate do investimento. 

\item Datas-focais intermediárias:
    Datas localizadas entre a data-pólo inicial e a data-pólo final, onde existem movimentações financeiras. 

\item Diagrama de tempo:
    Diagrama que apresenta o fluxo de caixa juntamente com os instantes de tempo das movimentações. Por convenção, recebimentos, entradas ou fluxos positivos são representados por setas apontando para cima ao passo que saídas ou pagamentos ou fluxos negativos são representados por setas apontando para baixo. 
\end{itemize}

\textbf{F}
\begin{itemize}
 \item Fluxo de Caixa:
    Montante de caixa recebido e gasto por uma entidade durante um período de tempo definido, algumas vezes ligado a um projeto específico. 
\end{itemize}

\textbf{J}
\begin{itemize}
 \item Juros de uma prestação em um sistema de amortização:
    Quantia paga além do valor amortizado em cada parcela. É calculada em função da taxa de juros e do saldo devedor do instante anterior. 

\item Juros Simples:
    No sistema de juros simples, os juros são calculados sobre o principal da dívida (valor inicial emprestado ou aplicado) 

\item Juros Compostos:
    No sistema de juros compostos, os juros de cada período somam-se à dívida, incidindo juros sobre ele no período seguinte 
\end{itemize}

\textbf{P}
\begin{itemize}
 \item Pagamento Antecipado:
Pagamento de uma prestação que é realizado no início de um dado período.

\item Pagamento Postecipado:
Pagamento de uma prestação que é realizado no término de um dado período.

\item Prestação:
    Instrumento de recuperação ou de devolução de um capital acrescido dos juros contratados no período de negociação da transação. 
\end{itemize}

\textbf{S}
\begin{itemize}
 \item Saldo Devedor:
    Montante que ainda necessita ser abatido em um dado instante de tempo considerando um certo sistema de amortização. 

\item Série Simples:
    Quando um fluxo de caixa apresenta apenas datas-pólo, ou seja, temos apenas dois fluxos. 

\item Série Complexa:
    Quando um fluxo apresenta ao menos uma data-focal intermediária além das datas-pólo, ou seja, tem-se no mínimo três fluxos. 

\item Série Uniforme de Prestações Uniformes:
    Conjunto de pagamentos ou recebimentos nominais iguais, dispostos em períodos constantes ao longo de uma série complexa de fluxo de caixa. Pode ser de três tipos: postecipada, antecipada ou diferida. 

\item Série Uniforme de Prestações Uniformes Antecipada:
    São pagamentos ou recebimentos distribuídos por todos os períodos de uma série complexa de fluxo de caixa, inclusive na data-pólo inicial ou data-focal zero. 

\item Série Uniforme de Prestações Uniformes Diferida:
    São pagamentos ou recebimentos distribuídos em uma série complexa de fluxo de caixa por períodos situados após um determinado prazo de carência. 

\item Série Uniforme de Prestações Uniformes Postecipada:
    São pagamentos ou recebimentos distribuídos por todos os períodos de uma série complexa de fluxo de Caixa, menos na data-pólo inicial ou data-focal zero. 

\item Sistema de Amortização:
    É a forma pela qual são calculadas as prestações que você vai pagar no decorrer do financiamento. Nos empréstimos de longo prazo, esses pagamentos são, normalmente, efetuados por parcelas (prestações), compostas de cotas de amortização e de juros. 

\item Sistema de Amortização Constante:
    É um método onde as amortizações do capital apresentam comportamento uniforme ou constante durante o período de vigência de uma operação financeira. Os juros apresentam valores heterogêneos e decrescentes ao longo do tempo, o que resulta em prestações heterogêneas e decrescentes ao longo do tempo. 

\item Sistema de Amortização Mista:
    Criado em 1979 pelo BNH. Representa um plano de amortização misto, a partir da combinação entre os sistemas Francês e Constante, onde juros, amortização e prestação, derivam de médias aritméticas envolvendo os valores calculados no PRICE e no SAC. Os juros apresentam valores heterogêneos e decrescentes ao longo do tempo. Com os juros heterogêneos e amortizações homogêneas temos uma configuração de prestações heterogêneas e decrescentes. 

\item Sistema Francês:
    Criado no século XVIII pelo matemático, filósofo e teólogo inglês Richard Price (por isso o nome Sistema Price), é um método de amortização com prestações iguais e vencidas, onde juros e amortizações portam-se de maneiras decrescente e crescente. 
\end{itemize}

\textbf{T}
\begin{itemize}
\item Taxa de Juros:
    É o percentual que remunera o capital. Uma taxa de juros incide sobre um capital disposto em um fluxo de caixa sempre quando da ocorrência de uma variação temporal, conhecida como freqüência intervalar necessária à formação dos juros 
\end{itemize}

\textbf{V}
\begin{itemize}
 \item Valor Atual ou Valor Presente:
    Representa um fluxo qualquer situado na data-pólo inicial. 

\item Valor Nominal
    Representa um fluxo qualquer situado entre a data-focal intermediária 1 e a data-pólo final. 

\item Valor Futuro Antecipado:
    Representa o somatório de todas as prestações capitalizadas à data-pólo final, com uma mesma taxa de juros; o PMT contido na data-pólo final não sofre nenhuma alteração 

\item Valor Futuro Postecipado:
    Representa o somatório de todas as prestações capitalizadas à data-pólo final, 
com uma mesma taxa de juros.

\item Valor Presente Antecipado:
    Representa o somatório de todas as prestações descontadas à data-pólo inicial, com uma mesma taxa de juros; o PMT data-pólo inicial não sofre nenhuma alteração 

\item Valor Presente Postecipado:
    Representa o somatório de todas as prestações descontadas à data-pólo inicial, com uma mesma taxa de juros. 
\end{itemize}